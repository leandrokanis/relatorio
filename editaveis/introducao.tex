\chapter*[Introdução]{Introdução}
\addcontentsline{toc}{chapter}{Introdução}

Atualmente, com o crescimento da tecnologia  e com o medo de um dia o petróleo não suprir toda a demanda de energia do mundo, surgiu uma forte procura por alternativas que possam substituir essa matéria-prima que move grandes economias no mundo. Uma dessas alternativas é a energia eólica, que é obtida por meio da conversão de energia cinética de translação em energia cinética de rotação. No Brasil, essa fonte é uma das formas renováveis que mais cresce atualmente por ser um país continental e ter locais com ventos contínuos propícios à sua utilização.
        	Uma forma de diversão de crianças e adultos é o hábito de soltar pipa , que os historiadores dizem que nasceu há mais de 1200 a.C na China , em que foi uma forma inspiradora para cientistas italianos pensarem nesse simples brinquedo como uma fonte de energia eólica.  A Kitegen foi uma precursora nessa área de pipas para gerar energia eólica e foi uma inspiração para estudantes do MIT (Instituto de Tecnologia de Massachusetts) criarem uma pipa eólica para ser testada no Alasca e no Canadá, que acabou atraindo olhares de investidores. Esta empresa tentou mudar a ideia de que apenas existiria energia eólica por pás, batendo na tecla que poderia haver a possibilidade de uma geração de energia eólica em altitudes entre 300 e 600 metros, onde há ventos que chegam a ser entre 5 a 8 vezes mais fortes e constantes do que os ventos ao nível do mar.