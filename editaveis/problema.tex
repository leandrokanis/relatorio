\chapter[Descrição do Problema]{Descrição do Problema}
\addcontentsline{toc}{chapter}{Descrição do Problema}

O Brasil é um país muito extenso e dotado de diversos recursos naturais. No entanto, cerca de 90\% da energia é produzida nas hidrelétricas, sendo esta energia  resultado da água localizada em níveis adequados no reservatório. O problema é que a ausência de chuva prejudica a oferta de energia, deixando a população com o risco de abastecimento . E o que se pode constatar é que mesmo com tamanha grandeza e recursos naturais, até hoje existem comunidades no Brasil que não possuem acesso à energia elétrica.
O aumento no consumo de energia elétrica por parte da população e a falta de investimento nas hidrelétricas fazem com que o nosso país esteja à beira de um caos. A crise energética está evidenciando vários problemas, até então, tratados com indiferença pelo poder público, que são: a iminente crise de água (resultado da super exploração e falta de preocupação ambiental com os mananciais), a má distribuição de água, desmatamento, desperdício e conflitos de uso . Estes são alguns dos problemas que tornam a escassez de água um problema cada vez mais iminente. Essa situação é resultado de um padrão de desenvolvimento sem planejamento, que consome muita água, energia e não protege os mananciais.
Devido aos problemas relatados acima, é necessário pensar em utilizar outras formas alternativas de energia, e que é preciso diversificar a produção de energia. A energia eólica seria uma alternativa para “desafogar” um pouco a produção de energia recebida em grande parte das hidrelétricas, pois é um tipo de energia limpa, sustentável e principalmente inesgotável. É uma boa alternativa para o Brasil, pois o país tem um vasto potencial para a energia advinda dos ventos, facilitando em muitos casos, o acesso a algumas comunidades isoladas. 
