\chapter*[Justificativa]{Justificativa}
\addcontentsline{toc}{chapter}{Justificativa}

Baseados na necessidade de se utilizar uma fonte de energia  que fosse capaz de representar uma engenhosa solução para a falta ou a má distribuição de energia elétrica em localidades isoladas do nosso país, percebeu-se a necessidade do desenvolvimento de um novo produto. Este, tem interesse em representar uma alternativa para a falta de energia em pequenas cidades que hoje não são contempladas com esse bem indispensável, através de uma usina eólica de fácil e rápida instalação, que possa gerar energia elétrica até mesmo em lugares que não possuem muito vento.
O projeto tem como diferencial, a possibilidade de atuação em áreas que outras usinas eólicas não possam atuar ou não consigam desempenhar resultados tão bons, pois as mesmas dependem de grandes espaços para instalação de aparelhos, grandes volumes de ventos e muito capital investido. Já o produto projetado, se baseia em um balão inflado que atua a cerca de 300 metros de altitude, onde os ventos são mais fortes e constantes. Além disso, por utilizar materiais baratos e possuir uma grande eficiência até mesmo com modelos bem menores que seus concorrentes, o mesmo pode atuar em espaços menores e menos favorecidos por correntes eólicas, tudo isso com custos baixos e fácil instalação.