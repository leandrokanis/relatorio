\chapter[Conclusão]{Conclusão}

O projeto “Geração de Energia usando Pipas” teve como principal objetivo e foco, através de estudo feito sobre tipo da pipa, local de atuação da pipa e alvo atingido pela pipa, de gerar energia elétrica, advinda de energia eólica da pipa, para a comunidade de Betânia do Piauí, localizada no Piauí, que mais necessita de ajuda. Com isso, através dos dados obtidos, foi possível construir o protótipo inicial do Sistema de forma a gerar energia elétrica com uso de pipas infláveis para até 200 casas da comunidade. Tudo foi estudado e planejado (em que foi realizado um orçamento inicial) para construir a primeira etapa do projeto.

O local escolhido da comunidade Betânia do Piauí foi alterado na Sprint 2, pelo fato dessa comunidade de zona rural ser isolada energeticamente e por necessitar urgentemente de uma boa qualidade de vida. Antes, o local da Sprint 1 era perto da usina hidrelétrica do Lago Paranoá, localizada em Brasília - Distrito Federal, em que ele foi alterado pelo fato da usina já atender energeticamente toda a população em volta do Lago e de Brasília.
Quanto à transmissão, conclui-se que esta se dará por meio de cabos ligados aos cabos de ancoragem da pipa, os geradores internos fazem com que a eletricidade captada seja passada aos transformadores em terra.

Com relação ao armazenamento, dentre as variedades de sistemas, o mais conveniente é a bateria que transforma a energia mecânica em energia elétrica na forma de corrente contínua e é o que mais se adequa para atender médias potências, abastecendo um aglomerado populacional pequeno.

Comparando as características de cada microcontrolador apresentado no Datasheet de cada um, o dispositivo mais viável seria o arduíno, pois ele apresenta caraterísticas já suficientes para nosso projeto, uma linguagem de programação bem intuitiva de programa, custo baixo, e sem necessitar a montagem de placa com periféricos como o PIC16. Já com o PIC16, a dificuldade seria implementar uma placa com os periféricos suficientes, e a Spartan3 tem um preço muito elevado.

O sensor escolhido para fazer parte do projeto da PIPA foi o da serie AP-V80 modelo AP-16S da marca KEYENCE. Onde a estrutura do cabo feita de aço inoxidável e detecta multifluidos como ar e Óleos. Trabalha de -20 ate 100 graus sem congelar e é resistente a pressão de ate 75 Mpa.
Dentre os materiais pesquisados, o poliestireno de baixa densidade atendeu às exigências feitas com relação ao modelo da pipa e a custos menores.

Por fim, conclui-se que o projeto possui um bom custo benefício para atender a comunidade que mais necessita de ajuda e atenção, não só energeticamente, mas eticamente também, pois todo ser humano precisa ter boas condições de vida .